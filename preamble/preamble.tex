\usepackage{lipsum}

\setlength{\headheight}{14.49998pt} % Fix warning.

%%%%%%%%%%%%%%%%%%%%%%%%%%%%%%%%%%%%%%%%
%% Other packages
%%%%%%%%%%%%%%%%%%%%%%%%%%%%%%%%%%%%%%%%
\usepackage{amssymb}
\usepackage{mathtools}
\usepackage{mathrsfs}
\usepackage{subcaption}
\usepackage{url}

% \usepackage{showframe} % DEBUG

\usepackage{wrapfig}
\usepackage{xcolor}
\usepackage{graphicx}
\graphicspath{{figures/}}
\usepackage{amsmath}
\allowdisplaybreaks
\usepackage{amsfonts}
\usepackage{amsthm}
\newtheorem{theorem}{Theorem}
\usepackage{cleveref}
\newtheorem{proposition}{Proposition}
\crefname{proposition}{proposition}{propositions}
\crefname{theorem}{theorem}{theorems}
\crefname{equation}{equation}{equations}
\crefname{section}{section}{sections}
\crefname{chapter}{chapter}{chapters}
\crefname{part}{part}{parts}
\crefname{figure}{figure}{figures}
\crefname{table}{table}{tables}
\usepackage{svg}
\usepackage{caption}
\usepackage{adjustbox}
\usepackage{rotating}
\usepackage{longtable,tabularx,booktabs}
\usepackage[flushleft]{threeparttable}
\usepackage{multirow}
\usepackage{lipsum}
\usepackage{makecell}
\usepackage{mdframed}
\usepackage{framed}
\usepackage{comment}
\usepackage{ragged2e}
\usepackage{soul}
\usepackage{xparse}
\usepackage{dsfont}
\usepackage{colortbl}
\makeatletter
\@namedef{ver@fixltx2e.sty}{}
\makeatother
\usepackage{dblfloatfix}
\usepackage{hyphenat}
\usepackage{bookmark}

\hypersetup{hidelinks}

\hyphenation{ana-lysis}
\hyphenation{Con-strained-Zero}

\definecolor{pastelMagenta}{HTML}{FF48CF}
\definecolor{pastelPurple}{HTML}{8770FE}
\definecolor{pastelBlue}{HTML}{1BA1EA}
\definecolor{pastelSeaGreen}{HTML}{14B57F}
\definecolor{pastelGreen}{HTML}{3EAA0D}
\definecolor{pastelOrange}{HTML}{C38D09}
\definecolor{pastelRed}{HTML}{F5615C}
\definecolor{julia_blue}{HTML}{4063D8}
\definecolor{julia_green}{HTML}{389826}
\definecolor{julia_purple}{HTML}{9558B2}
\definecolor{julia_red}{HTML}{CB3C33}

\definecolor{juliafunccolor}{HTML}{2e51a2}
\definecolor{juliaparenscolor}{HTML}{f2d71c}

\newcommand\blfootnote[1]{%
  \begingroup
  \renewcommand\thefootnote{}\footnote{#1}%
  \addtocounter{footnote}{-1}%
  \endgroup
}

\NewDocumentCommand{\chapterquote}{ m m o }{%
  \begin{flushright}
    \begin{minipage}{\IfValueTF{#3}{#3}{0.52\linewidth}}
      \RaggedRight
      \textit{#1}

      \rule{\linewidth}{0.4pt}

      \hfill #2
    \end{minipage}
  \end{flushright}
  \vspace{\baselineskip}
}

%%%%%%%%%%%%%%%%%%%%%%%%%%%%%%%%%%%%%%%%
% Algorithms
%%%%%%%%%%%%%%%%%%%%%%%%%%%%%%%%%%%%%%%%
\usepackage{algorithmicx}
\usepackage{algorithm}
\usepackage{algpseudocode}

% http://tug.ctan.org/tex-archive/macros/latex/contrib/algorithmicx/algpseudocode.sty
\algdef{SE}[FOR]{ParallelFor}{EndParallelFor}[1]{\textbf{parallel} \algorithmicfor\ #1}{\algorithmicend}%
\algdef{SE}[FOR]{For}{EndFor}[1]{\algorithmicfor\ #1}{\algorithmicend}%
\algdef{SE}[IF]{If}{EndIf}[1]{\algorithmicif\ #1}{\algorithmicend}%
\algdef{SE}[IF]{NewIf}{NewEndIf}[1]{\new{\algorithmicif}\ #1}{\algorithmicend}%
\algdef{C}[IF]{IF}{ElsIf}[1]{\algorithmicelse\ \algorithmicif\ #1}%
\algdef{Ce}[ELSE]{IF}{Else}{EndIf}{\algorithmicelse}%
\algdef{Ce}[ELSE]{IF}{Else}{NewEndIf}{\algorithmicelse}%
\algdef{SE}[FUNCTION]{Function}{EndFunction}%
   [2]{\algorithmicfunction\ \textproc{#1}\ifthenelse{\equal{#2}{}}{}{(#2)}}%
   {\algorithmicend}%
\algtext*{EndFunction} % remove "end" for functions only
\algrenewcommand\alglinenumber[1]{\color{gray}\tiny #1}

\algtext*{EndFunction} % remove "end" for functions only
\algtext*{EndFor} % remove "end" for for-loops only
\algtext*{EndParallelFor} % remove "end" for parallel for-loops only
\algtext*{EndIf} % remove "end" for if-statements only
\algtext*{NewEndIf} % remove "end" for (new) if-statements only

\definecolor{commentgray}{rgb}{0.6, 0.6, 0.6}
\newcommand{\GrayComment}[1]{{\hfill{\color{commentgray}$\triangleright$ #1}}}

\newcommand{\AlignedComment}[2][0.65\linewidth]{%
  \leavevmode\hfill\makebox[#1][l]{\quad{\color{lightgray}\(\triangleright\)~#2}}}

\usepackage{newfloat}
\usepackage{listings}
\DeclareCaptionStyle{ruled}{labelfont=normalfont,labelsep=colon,strut=off}
\lstset{%
	basicstyle={\footnotesize\ttfamily},%
	numbers=left,numberstyle=\footnotesize,xleftmargin=2em,%
	aboveskip=0pt,belowskip=0pt,%
	showstringspaces=false,tabsize=2,breaklines=true}
\floatstyle{ruled}
\newfloat{listing}{tb}{lst}{}
\floatname{listing}{Listing}

% Hack to remove white lines (as seen in the second case), problem with \footnotesize
% https://tex.stackexchange.com/a/129651
\newenvironment{juliaframe}{%
    \linespread{1.25}\selectfont
    \begin{mdframed}[%
        backgroundcolor=backgroundgray,%
        hidealllines=true,%
        innerleftmargin=0pt,%
        innertopmargin=-1pt,%
        innerbottommargin=-6pt,%
        innerrightmargin=0pt,
        skipabove=12pt,
        skipbelow=-6pt,
    ]
}{%
    \end{mdframed}
}

\input{preamble/julia_listings}

\lstdefinelanguage{JuliaLocal}{
    language = Julia, % inherit Julia lang. to add keywords
    morekeywords = [3]{reset, initialize, generate_input, evaluate, bayesian_safety_validation, falsification, most_likely_failure, p_estimate}, % define more functions
    morekeywords = [2]{BayesianSafetyValidation, SystemParameters, Input, RunwayDetectionSystemParameters, OperationalParameters, TruncatedNormal, Normal}, % define more types and modules
}


%%%%%%%%%%%%%%%%%%%%%%%%%%%%%%%%%%%%%%%%
%% Formatting Helpers
%%%%%%%%%%%%%%%%%%%%%%%%%%%%%%%%%%%%%%%%
\newcommand{\rulesep}{\unskip\ \textcolor{lightgray}{\vrule}\ }
\newcommand{\tworow}[1]{\multirow{2}{*}{{#1}}}
\newcommand{\cpomcpow}{\cite{sunberg2018online,wu2021adaptive}}
\newcommand{\cdespot}{\cite{ye2017despot,wu2021adaptive}}
\newcommand{\cadaops}{\cite{wu2021adaptive}}
\newcommand{\scs}[1]{{\scriptsize#1}}
\newcommand{\pp}{\phantom{)}}
\newcommand{\td}[1]{\color{magenta}#1}
\newcommand{\negphantom}[1]{\settowidth{\dimen0}{#1}\hspace*{-\dimen0}}
\newcommand{\np}{\negphantom{)}}
\newcommand{\lit}[1]{{\color[rgb]{0.7,0.7,0.7}(#1)}\np}
\newcommand{\litdesc}[1]{\lit{#1}\phantom{)}}
\newcommand{\textdoublequotes}{\textsf{''}}
\newcommand{\sameresultsparen}{\textdoublequotes{}\phantom{\ 0.00)}}
\newcommand{\sameresults}{\textdoublequotes{}}


%%%%%%%%%%%%%%%%%%%%%%%%%%%%%%%%%%%%%%%%
%% TiKZ, PGF plots, and colors
%%%%%%%%%%%%%%%%%%%%%%%%%%%%%%%%%%%%%%%%
\usepackage{tikz}
\usepackage{pgfplots}
\pgfplotsset{compat=1.17}

\usetikzlibrary{calc}
\usetikzlibrary{shapes.geometric}
\usetikzlibrary{external}
\usetikzlibrary{patterns}
\usetikzlibrary{shapes,arrows,fit}
\usetikzlibrary{positioning}
\usetikzlibrary{arrows.meta, calc, shapes}
\usetikzlibrary{graphs}
\usetikzlibrary{decorations.pathmorphing}
\usetikzlibrary{decorations.pathreplacing}
\usetikzlibrary{backgrounds}
\usetikzlibrary{shadows}

\definecolor{blues1}{RGB}{198, 219, 239}
\definecolor{blues2}{RGB}{158, 202, 225}
\definecolor{blues3}{RGB}{107, 174, 214}
\definecolor{blues4}{RGB}{49, 130, 189}
\definecolor{blues5}{RGB}{8, 81, 156}

\definecolor{grays1}{RGB}{219, 219, 219}
\definecolor{grays2}{RGB}{202, 202, 202}
\definecolor{grays3}{RGB}{174, 174, 174}
\definecolor{grays4}{RGB}{130, 130, 130}
\definecolor{grays5}{RGB}{81, 81, 81}

\definecolor{gray1}{HTML}{999999}
\definecolor{gray2}{HTML}{CCCCCC}
\definecolor{gray3}{HTML}{D9D9D9}
\definecolor{gray4}{HTML}{EFEFEF}

\definecolor{darkgreen}{HTML}{38761D}
\definecolor{darkred}{HTML}{980000}
\definecolor{cardinal}{HTML}{B83A4B}

\definecolor{newcolor}{HTML}{C70039}
\definecolor{tablecolorgood}{HTML}{007662}
\definecolor{tablecolorbad}{HTML}{8c1515}
\definecolor{backgroundgray}{HTML}{F2F2F2}

\definecolor{timingcolor}{rgb}{0.55, 0.55, 0.55}
\newcommand{\tcolor}[1]{\color{timingcolor}#1}

% Standard colors
\def\primarycolor{black}
\def\secondarycolor{white}

%% PGFPlotsX.jl
\usetikzlibrary{arrows.meta}
\usetikzlibrary{backgrounds}
\usepgfplotslibrary{patchplots}
\usepgfplotslibrary{fillbetween}
\pgfplotsset{%
    layers/standard/.define layer set={%
        background,axis background,axis grid,axis ticks,axis lines,axis tick labels,pre main,main,axis descriptions,axis foreground%
    }{
        grid style={/pgfplots/on layer=axis grid},%
        tick style={/pgfplots/on layer=axis ticks},%
        axis line style={/pgfplots/on layer=axis lines},%
        label style={/pgfplots/on layer=axis descriptions},%
        legend style={/pgfplots/on layer=axis descriptions},%
        title style={/pgfplots/on layer=axis descriptions},%
        colorbar style={/pgfplots/on layer=axis descriptions},%
        ticklabel style={/pgfplots/on layer=axis tick labels},%
        axis background@ style={/pgfplots/on layer=axis background},%
        3d box foreground style={/pgfplots/on layer=axis foreground},%
    },
}


%%%%%%%%%%%%%%%%%%%%%%%%%%%%%%%%%%%%%%%%
%% Math
%%%%%%%%%%%%%%%%%%%%%%%%%%%%%%%%%%%%%%%%
\DeclareMathOperator{\Var}{Var}
\newcommand*{\defeq}{\stackrel{\text{def}}{=}}
\newcommand{\bpi}{\boldsymbol\pi}
\newcommand*\diff{\mathop{}\!\mathrm{d}}

\definecolor{resultcolor}{HTML}{000000}

\newsavebox\CBox
\def\mathBF#1{\sbox\CBox{$#1$}\resizebox{\wd\CBox}{\ht\CBox}{{\color{resultcolor}$\mathbf{#1}$}}}

\usepackage{siunitx}
\sisetup{detect-weight=true,detect-family=true,group-digits=false,separate-uncertainty,group-separator={,}}
\newcommand{\pmm}{{\normalfont$\pm$}}

% https://tex.stackexchange.com/questions/239242/why-does-big-mid-not-work
\makeatletter
\let\amsmath@bigmidtmp\bigm

\newcommand{\bigmid}[1]{%
  \ifcsname fenced@\string#1\endcsname
    \expandafter\@firstoftwo
  \else
    \expandafter\@secondoftwo
  \fi
  {\expandafter\amsmath@bigmidtmp\csname fenced@\string#1\endcsname}%
  {\amsmath@bigmidtmp#1}%
}
\newcommand{\DeclareFence}[2]{\@namedef{fenced@\string#1}{#2}}
\makeatother

\DeclareFence{\mid}{|}

% https://mirror.mwt.me/ctan/fonts/mathabx/texinputs/mathabx.sty
\DeclareFontFamily{U}{mathx}{\hyphenchar\font45}
\DeclareFontShape{U}{mathx}{m}{n}{
      <5> <6> <7> <8> <9> <10>
      <10.95> <12> <14.4> <17.28> <20.74> <24.88>
      mathx10
      }{}
\DeclareSymbolFont{mathx}{U}{mathx}{m}{n}
\DeclareFontSubstitution{U}{mathx}{m}{n}

% https://mirror.math.princeton.edu/pub/CTAN/fonts/mathabx/texinputs/mathabx.dcl
\DeclareMathSymbol{\betterbigvee}         {1}{mathx}{"9A}
\DeclareMathSymbol{\betterbigwedge}       {1}{mathx}{"99}

\DeclareMathAccent{\widebar}{0}{mathx}{"73} % from mathabx

\newcommand{\overarrow}[2]{\overset{\mathclap{\substack{#2 \\ \downarrow}}}{#1}}
\newcommand{\underarrow}[2]{\underset{\mathclap{\substack{\uparrow \\ #2}}}{#1}}

\let\oldvec\vec
\renewcommand{\vec}[1]{\vect{#1}}


%%%%%%%%%%%%%%%%%%%%%%%%%%%%%%%%%%%%%%%%
%% Commands
%%%%%%%%%%%%%%%%%%%%%%%%%%%%%%%%%%%%%%%%
\newcommand{\note}[1]{\textbf{\textcolor{cyan}{#1}}}
\newcommand{\citeneeded}{\textbf{\textcolor{magenta}{[CITE]}}}
\newcommand{\colorurl}[1]{\textcolor{cardinal}{\url{#1}}}

\newcommand{\new}[1]{{\color{newcolor}#1}}
\newcommand{\bpm}{\boldsymbol\pm}

% Switch between regular (symbol) and numeric footnotes
\renewcommand{\thefootnote}{\fnsymbol{footnote}}
\newcommand{\numericfootnote}[1]{%
\let\oldthefootnote=\thefootnote%
\stepcounter{mpfootnote}%
\addtocounter{footnote}{-1}%
\renewcommand{\thefootnote}{\arabic{mpfootnote}}%
\footnote{#1}%
\let\thefootnote=\oldthefootnote%
}

\setlength{\fboxsep}{0pt}
\setlength{\fboxrule}{0.3pt}
