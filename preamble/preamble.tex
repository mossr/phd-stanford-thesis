\usepackage{lipsum}

\setlength{\headheight}{14.49998pt} % Fix warning.

%%%%%%%%%%%%%%%%%%%%%%%%%%%%%%%%%%%%%%%%
%% Other packages
%%%%%%%%%%%%%%%%%%%%%%%%%%%%%%%%%%%%%%%%
\usepackage{amssymb}
\usepackage{mathtools}
\usepackage{mathrsfs}
\usepackage{subcaption}
\usepackage{url}

% \usepackage{showframe} % DEBUG

\usepackage{wrapfig}
\usepackage{xcolor}
\usepackage{graphicx}
\graphicspath{{figures/}}
\usepackage{amsmath}
\allowdisplaybreaks
\usepackage{amsfonts}
\usepackage{amsthm}
\newtheorem{theorem}{Theorem}
\usepackage{cleveref}
\newtheorem{proposition}{Proposition}
\crefname{proposition}{proposition}{propositions}
\crefname{theorem}{theorem}{theorems}
\crefname{equation}{equation}{equations}
\crefname{section}{section}{sections}
\crefname{chapter}{chapter}{chapters}
\crefname{part}{part}{parts}
\crefname{figure}{figure}{figures}
\crefname{table}{table}{tables}
\usepackage{svg}
\usepackage{caption}
\usepackage{adjustbox}
\usepackage{rotating}
\usepackage{longtable,tabularx,booktabs}
\usepackage[flushleft]{threeparttable}
\usepackage{multirow}
\usepackage{lipsum}
\usepackage{makecell}
\usepackage{mdframed}
\usepackage{framed}
\usepackage{comment}
\usepackage{ragged2e}
\usepackage{soul}
\usepackage{xparse}
\usepackage{dsfont}
\usepackage{colortbl}
\makeatletter
\@namedef{ver@fixltx2e.sty}{}
\makeatother
\usepackage{dblfloatfix}
\usepackage{hyphenat}
\usepackage{bookmark}

\hypersetup{hidelinks}

\hyphenation{ana-lysis}
\hyphenation{Con-strained-Zero}

\definecolor{pastelMagenta}{HTML}{FF48CF}
\definecolor{pastelPurple}{HTML}{8770FE}
\definecolor{pastelBlue}{HTML}{1BA1EA}
\definecolor{pastelSeaGreen}{HTML}{14B57F}
\definecolor{pastelGreen}{HTML}{3EAA0D}
\definecolor{pastelOrange}{HTML}{C38D09}
\definecolor{pastelRed}{HTML}{F5615C}
\definecolor{julia_blue}{HTML}{4063D8}
\definecolor{julia_green}{HTML}{389826}
\definecolor{julia_purple}{HTML}{9558B2}
\definecolor{julia_red}{HTML}{CB3C33}

\definecolor{juliafunccolor}{HTML}{2e51a2}
\definecolor{juliaparenscolor}{HTML}{f2d71c}

\newcommand\blfootnote[1]{%
  \begingroup
  \renewcommand\thefootnote{}\footnote{#1}%
  \addtocounter{footnote}{-1}%
  \endgroup
}

\NewDocumentCommand{\chapterquote}{ m m o }{%
  \begin{flushright}
    \begin{minipage}{\IfValueTF{#3}{#3}{0.52\linewidth}}
      \RaggedRight
      \textit{#1}

      \rule{\linewidth}{0.4pt}

      \hfill #2
    \end{minipage}
  \end{flushright}
  \vspace{\baselineskip}
}

%%%%%%%%%%%%%%%%%%%%%%%%%%%%%%%%%%%%%%%%
% Algorithms
%%%%%%%%%%%%%%%%%%%%%%%%%%%%%%%%%%%%%%%%
\usepackage{algorithmicx}
\usepackage{algorithm}
\usepackage{algpseudocode}

% http://tug.ctan.org/tex-archive/macros/latex/contrib/algorithmicx/algpseudocode.sty
\algdef{SE}[FOR]{ParallelFor}{EndParallelFor}[1]{\textbf{parallel} \algorithmicfor\ #1}{\algorithmicend}%
\algdef{SE}[FOR]{For}{EndFor}[1]{\algorithmicfor\ #1}{\algorithmicend}%
\algdef{SE}[IF]{If}{EndIf}[1]{\algorithmicif\ #1}{\algorithmicend}%
\algdef{SE}[IF]{NewIf}{NewEndIf}[1]{\new{\algorithmicif}\ #1}{\algorithmicend}%
\algdef{C}[IF]{IF}{ElsIf}[1]{\algorithmicelse\ \algorithmicif\ #1}%
\algdef{Ce}[ELSE]{IF}{Else}{EndIf}{\algorithmicelse}%
\algdef{Ce}[ELSE]{IF}{Else}{NewEndIf}{\algorithmicelse}%
\algdef{SE}[FUNCTION]{Function}{EndFunction}%
   [2]{\algorithmicfunction\ \textproc{#1}\ifthenelse{\equal{#2}{}}{}{(#2)}}%
   {\algorithmicend}%
\algtext*{EndFunction} % remove "end" for functions only
\algrenewcommand\alglinenumber[1]{\color{gray}\tiny #1}

\algtext*{EndFunction} % remove "end" for functions only
\algtext*{EndFor} % remove "end" for for-loops only
\algtext*{EndParallelFor} % remove "end" for parallel for-loops only
\algtext*{EndIf} % remove "end" for if-statements only
\algtext*{NewEndIf} % remove "end" for (new) if-statements only

\definecolor{commentgray}{rgb}{0.6, 0.6, 0.6}
\newcommand{\GrayComment}[1]{{\hfill{\color{commentgray}$\triangleright$ #1}}}

\newcommand{\AlignedComment}[2][0.65\linewidth]{%
  \leavevmode\hfill\makebox[#1][l]{\quad{\color{lightgray}\(\triangleright\)~#2}}}

\usepackage{newfloat}
\usepackage{listings}
\DeclareCaptionStyle{ruled}{labelfont=normalfont,labelsep=colon,strut=off}
\lstset{%
	basicstyle={\footnotesize\ttfamily},%
	numbers=left,numberstyle=\footnotesize,xleftmargin=2em,%
	aboveskip=0pt,belowskip=0pt,%
	showstringspaces=false,tabsize=2,breaklines=true}
\floatstyle{ruled}
\newfloat{listing}{tb}{lst}{}
\floatname{listing}{Listing}

% Hack to remove white lines (as seen in the second case), problem with \footnotesize
% https://tex.stackexchange.com/a/129651
\newenvironment{juliaframe}{%
    \linespread{1.25}\selectfont
    \begin{mdframed}[%
        backgroundcolor=backgroundgray,%
        hidealllines=true,%
        innerleftmargin=0pt,%
        innertopmargin=-1pt,%
        innerbottommargin=-6pt,%
        innerrightmargin=0pt,
        skipabove=12pt,
        skipbelow=-6pt,
    ]
}{%
    \end{mdframed}
}

\usepackage{listings}
\usepackage{xcolor}

\lstdefinelanguage{Julia}{
    % functions
    keywords=[3]{abs,abs2,abspath,accept,accumulate,accumulate!,acos,acos_fast,acosd,acosh,acosh_fast,acot,acotd,acoth,acsc,acscd,acsch,adjoint,adjoint!,all,all!,allunique,angle,angle_fast,any,any!,append!,apropos,ascii,asec,asecd,asech,asin,asin_fast,asind,asinh,asinh_fast,assert,asyncmap,asyncmap!,atan,atan2,atan2_fast,atan_fast,atand,atanh,atanh_fast,atexit,atreplinit,axes,backtrace,base,basename,beta,big,bin,bind,binomial,bitbroadcast,bitrand,bits,bitstring,bkfact,bkfact!,blkdiag,broadcast,broadcast!,broadcast_getindex,broadcast_setindex!,bswap,bytes2hex,cat,catch_backtrace,catch_stacktrace,cbrt,cbrt_fast,cd,ceil,cfunction,cglobal,charwidth,checkbounds,checkindex,chmod,chol,cholfact,cholfact!,chomp,chop,chown,chr2ind,circcopy!,circshift,circshift!,cis,cis_fast,clamp,clamp!,cld,clipboard,close,cmp,coalesce,code_llvm,code_lowered,code_native,code_typed,code_warntype,codeunit,codeunits,collect,colon,complex,cond,condskeel,conj,conj!,connect,consume,contains,convert,copy,copy!,copysign,copyto!,cor,cos,cos_fast,cosc,cosd,cosh,cosh_fast,cospi,cot,cotd,coth,count,count_ones,count_zeros,countlines,countnz,cov,cp,cross,csc,cscd,csch,ctime,ctranspose,ctranspose!,cummax,cummin,cumprod,cumprod!,cumsum,cumsum!,current_module,current_task,dec,deepcopy,deg2rad,delete!,deleteat!,den,denominator,deserialize,det,detach,diag,diagind,diagm,diff,digits,digits!,dirname,disable_sigint,display,displayable,displaysize,div,divrem,done,dot,download,dropzeros,dropzeros!,dump,eachcol,eachindex,eachline,eachmatch,edit,eig,eigfact,eigfact!,eigmax,eigmin,eigvals,eigvals!,eigvecs,eltype,empty,empty!,endof,endswith,enumerate,eof,eps,equalto,error,esc,escape_string,evalfile,exit,exp,exp10,exp10_fast,exp2,exp2_fast,exp_fast,expanduser,expm,expm!,expm1,expm1_fast,exponent,extrema,eye,factorial,factorize,falses,fd,fdio,fetch,fieldcount,fieldname,fieldnames,fieldoffset,filemode,filesize,fill,fill!,filter,filter!,finalize,finalizer,find,findfirst,findin,findlast,findmax,findmax!,findmin,findmin!,findn,findnext,findnz,findprev,first,fld,fld1,fldmod,fldmod1,flipbits!,flipdim,flipsign,float,floor,flush,fma,foldl,foldr,foreach,frexp,full,fullname,functionloc,gamma,gc,gc_enable,gcd,gcdx,gensym,get,get!,get_zero_subnormals,getaddrinfo,getalladdrinfo,gethostname,getindex,getipaddr,getkey,getnameinfo,getpeername,getpid,getsockname,givens,gperm,gradient,hash,haskey,hcat,hessfact,hessfact!,hex,hex2bytes,hex2bytes!,hex2num,homedir,htol,hton,hvcat,hypot,hypot_fast,identity,ifelse,ignorestatus,im,imag,in,include_dependency,include_string,ind2chr,ind2sub,indexin,indices,indmax,indmin,info,insert!,instances,intersect,intersect!,inv,invmod,invperm,invpermute!,ipermute!,ipermutedims,is,is_apple,is_bsd,is_linux,is_unix,is_windows,isabspath,isapprox,isascii,isassigned,isbits,isblockdev,ischardev,isconcrete,isconst,isdiag,isdir,isdirpath,isempty,isequal,iseven,isfifo,isfile,isfinite,ishermitian,isimag,isimmutable,isinf,isinteger,isinteractive,isleaftype,isless,isletter,islink,islocked,ismarked,ismatch,ismissing,ismount,isnan,isodd,isone,isopen,ispath,isperm,isposdef,isposdef!,ispow2,isqrt,isreadable,isreadonly,isready,isreal,issetgid,issetuid,issocket,issorted,issparse,issticky,issubnormal,issubset,issubtype,issymmetric,istaskdone,istaskstarted,istextmime,istril,istriu,isvalid,iswritable,iszero,join,joinpath,keys,keytype,kill,kron,last,lbeta,lcm,ldexp,ldltfact,ldltfact!,leading_ones,leading_zeros,length,less,lexcmp,lexless,lfact,lgamma,lgamma_fast,linearindices,linreg,linspace,listen,listenany,lock,log,log10,log10_fast,log1p,log1p_fast,log2,log2_fast,log_fast,logabsdet,logdet,logging,logm,logspace,lpad,lq,lqfact,lqfact!,lstat,lstrip,ltoh,lu,lufact,lufact!,lyap,macroexpand,map,map!,mapfoldl,mapfoldr,mapreduce,mapreducedim,mapslices,mark,match,matchall,max,max_fast,maxabs,maximum,maximum!,maxintfloat,mean,mean!,median,median!,merge,merge!,method_exists,methods,methodswith,middle,midpoints,mimewritable,min,min_fast,minabs,minimum,minimum!,minmax,minmax_fast,missing,mkdir,mkpath,mktemp,mktempdir,mod,mod1,mod2pi,modf,module_name,module_parent,mtime,muladd,mv,names,nb_available,ncodeunits,ndigits,ndims,next,nextfloat,nextind,nextpow,nextpow2,nextprod,nnz,nonzeros,norm,normalize,normalize!,normpath,notify,ntoh,ntuple,nullspace,num,num2hex,numerator,nzrange,object_id,occursin,oct,oftype,one,ones,oneunit,open,operm,ordschur,ordschur!,pairs,parent,parentindexes,parentindices,parse,partialsort,partialsort!,partialsortperm,partialsortperm!,peakflops,permute,permute!,permutedims,permutedims!,pi,pinv,pipeline,pointer,pointer_from_objref,pop!,popdisplay,popfirst!,position,pow_fast,powermod,precision,precompile,prepend!,prevfloat,prevind,prevpow,prevpow2,print,print_shortest,print_with_color,println,process_exited,process_running,prod,prod!,produce,promote,promote_rule,promote_shape,promote_type,push!,pushdisplay,pushfirst!,put!,pwd,qr,qrfact,qrfact!,quantile,quantile!,quit,rad2deg,rand,rand!,randcycle,randcycle!,randexp,randexp!,randjump,randn,randn!,randperm,randperm!,randstring,randsubseq,randsubseq!,range,rank,rationalize,read,read!,readandwrite,readavailable,readbytes!,readchomp,readdir,readline,readlines,readlink,readstring,readuntil,real,realmax,realmin,realpath,recv,recvfrom,redirect_stderr,redirect_stdin,redirect_stdout,redisplay,reduce,reducedim,reenable_sigint,reim,reinterpret,reload,relpath,rem,rem2pi,repeat,replace,replace!,repmat,repr,reprmime,reset,reshape,resize!,rethrow,retry,reverse,reverse!,reverseind,rm,rol,rol!,ror,ror!,rot180,rotl90,rotr90,round,rounding,rowvals,rpad,rsearch,rsearchindex,rsplit,rstrip,run,scale!,schedule,schur,schurfact,schurfact!,search,searchindex,searchsorted,searchsortedfirst,searchsortedlast,sec,secd,sech,seek,seekend,seekstart,select,select!,selectperm,selectperm!,send,serialize,set_zero_subnormals,setdiff,setdiff!,setenv,setindex!,setprecision,setrounding,shift!,show,showall,showcompact,showerror,shuffle,shuffle!,sign,signbit,signed,signif,significand,similar,sin,sin_fast,sinc,sincos,sind,sinh,sinh_fast,sinpi,size,sizehint!,sizeof,skip,skipchars,skipmissing,sleep,slicedim,sort,sort!,sortcols,sortperm,sortperm!,sortrows,sparse,sparsevec,spawn,spdiagm,speye,splice!,split,splitdir,splitdrive,splitext,spones,sprand,sprandn,sprint,spzeros,sqrt,sqrt_fast,sqrtm,squeeze,srand,stacktrace,start,startswith,stat,std,stdm,step,stride,strides,string,stringmime,strip,strwidth,sub2ind,subtypes,success,sum,sum!,sumabs,sumabs2,summary,supertype,svd,svdfact,svdfact!,svdvals,svdvals!,sylvester,symdiff,symdiff!,symlink,systemerror,take!,takebuf_array,takebuf_string,tan,tan_fast,tand,tanh,tanh_fast,task_local_storage,tempdir,tempname,thisind,tic,time,time_ns,timedwait,to_indices,toc,toq,touch,trace,trailing_ones,trailing_zeros,transcode,transpose,transpose!,tril,tril!,triu,triu!,trues,trunc,truncate,trylock,tryparse,typeintersect,typejoin,typemax,typemin,unescape_string,union,union!,unique,unique!,unlock,unmark,unsafe_copy!,unsafe_copyto!,unsafe_load,unsafe_pointer_to_objref,unsafe_read,unsafe_store!,unsafe_string,unsafe_trunc,unsafe_wrap,unsafe_write,unshift!,unsigned,uperm,valtype,values,var,varinfo,varm,vcat,vec,vecdot,vecnorm,versioninfo,view,wait,walkdir,warn,which,whos,widemul,widen,withenv,workspace,write,xor,yield,yieldto,zero,zeros,zip,applicable,eval,fieldtype,getfield,invoke,isa,isdefined,nfields,nothing,setfield!,throw,tuple,typeassert,typeof,uninitialized},%
    % module functions
    keywords=[3]{asum,axpby!,axpy!,blascopy!,dot,dotc,dotu,gbmv,gbmv!,gemm,gemm!,gemv,gemv!,ger!,hemm,hemm!,hemv,hemv!,her!,her2k,her2k!,herk,herk!,iamax,nrm2,sbmv,sbmv!,scal,scal!,symm,symm!,symv,symv!,syr!,syr2k,syr2k!,syrk,syrk!,trmm,trmm!,trmv,trmv!,trsm,trsm!,trsv,trsv!),abs,abs2,abspath,accept,accumulate,accumulate!,acos,acos_fast,acosd,acosh,acosh_fast,acot,acotd,acoth,acsc,acscd,acsch,adjoint,adjoint!,all,all!,allunique,angle,angle_fast,any,any!,append!,apropos,argmax,argmin,ascii,asec,asecd,asech,asin,asin_fast,asind,asinh,asinh_fast,assert,asyncmap,asyncmap!,atan,atan2,atan2_fast,atan_fast,atand,atanh,atanh_fast,atexit,atreplinit,axes,backtrace,base,basename,beta,bfft,bfft!,big,bin,bind,binomial,bitbroadcast,bitrand,bits,bitstring,bkfact,bkfact!,blkdiag,brfft,broadcast,broadcast!,broadcast_getindex,broadcast_setindex!,bswap,bytes2hex,cat,catch_backtrace,catch_stacktrace,cbrt,cbrt_fast,cd,ceil,cfunction,cglobal,charwidth,checkbounds,checkindex,chmod,chol,cholfact,cholfact!,chomp,chop,chown,chr2ind,circcopy!,circshift,circshift!,cis,cis_fast,clamp,clamp!,cld,clipboard,close,cmp,coalesce,code_llvm,code_lowered,code_native,code_typed,code_warntype,codeunit,codeunits,collect,colon,complex,cond,condskeel,conj,conj!,connect,consume,contains,conv,conv2,convert,copy,copy!,copysign,copyto!,cor,cos,cos_fast,cosc,cosd,cosh,cosh_fast,cospi,cot,cotd,coth,count,count_ones,count_zeros,countlines,countnz,cov,cp,cross,csc,cscd,csch,ctime,ctranspose,ctranspose!,cummax,cummin,cumprod,cumprod!,cumsum,cumsum!,current_module,current_task,dct,dct!,dec,deconv,deepcopy,deg2rad,delete!,deleteat!,den,denominator,deserialize,det,detach,diag,diagind,diagm,diff,digits,digits!,dirname,disable_sigint,display,displayable,displaysize,div,divrem,done,dot,download,dropzeros,dropzeros!,dump,eachcol,eachindex,eachline,eachmatch,edit,eig,eigfact,eigfact!,eigmax,eigmin,eigvals,eigvals!,eigvecs,eltype,empty,empty!,endof,endswith,enumerate,eof,eps,equalto,error,esc,escape_string,evalfile,exit,exp,exp10,exp10_fast,exp2,exp2_fast,exp_fast,expand,expanduser,expm,expm!,expm1,expm1_fast,exponent,extrema,eye,factorial,factorize,falses,fd,fdio,fetch,fft,fft!,fftshift,fieldcount,fieldname,fieldnames,fieldoffset,filemode,filesize,fill,fill!,filt,filt!,filter,filter!,finalize,finalizer,find,findfirst,findin,findlast,findmax,findmax!,findmin,findmin!,findn,findnext,findnz,findprev,first,fld,fld1,fldmod,fldmod1,flipbits!,flipdim,flipsign,float,floor,flush,fma,foldl,foldr,foreach,frexp,full,fullname,functionloc,gamma,gc,gc_enable,gcd,gcdx,gensym,get,get!,get_zero_subnormals,getaddrinfo,getalladdrinfo,gethostname,getindex,getipaddr,getkey,getnameinfo,getpeername,getpid,getsockname,givens,gperm,gradient,hash,haskey,hcat,hessfact,hessfact!,hex,hex2bytes,hex2bytes!,hex2num,homedir,htol,hton,hvcat,hypot,hypot_fast,idct,idct!,identity,ifelse,ifft,ifft!,ifftshift,ignorestatus,im,imag,in,include_dependency,include_string,ind2chr,ind2sub,indexin,indices,indmax,indmin,info,insert!,instances,intersect,intersect!,inv,invmod,invperm,invpermute!,ipermute!,ipermutedims,irfft,is,is_apple,is_bsd,is_linux,is_unix,is_windows,isabspath,isapprox,isascii,isassigned,isbits,isblockdev,ischardev,isconcrete,isconst,isdiag,isdir,isdirpath,isempty,isequal,iseven,isfifo,isfile,isfinite,ishermitian,isimag,isimmutable,isinf,isinteger,isinteractive,isleaftype,isless,isletter,islink,islocked,ismarked,ismatch,ismissing,ismount,isnan,isodd,isone,isopen,ispath,isperm,isposdef,isposdef!,ispow2,isqrt,isreadable,isreadonly,isready,isreal,issetgid,issetuid,issocket,issorted,issparse,issticky,issubnormal,issubset,issubtype,issymmetric,istaskdone,istaskstarted,istextmime,istril,istriu,isvalid,iswritable,iszero,join,joinpath,keys,keytype,kill,kron,last,lbeta,lcm,ldexp,ldltfact,ldltfact!,leading_ones,leading_zeros,length,less,lexcmp,lexless,lfact,lgamma,lgamma_fast,linearindices,linreg,linspace,listen,listenany,lock,log,log10,log10_fast,log1p,log1p_fast,log2,log2_fast,log_fast,logabsdet,logdet,logging,logm,logspace,lpad,lq,lqfact,lqfact!,lstat,lstrip,ltoh,lu,lufact,lufact!,lyap,macroexpand,map,map!,mapfoldl,mapfoldr,mapreduce,mapreducedim,mapslices,mark,match,matchall,max,max_fast,maxabs,maximum,maximum!,maxintfloat,mean,mean!,median,median!,merge,merge!,method_exists,methods,methodswith,middle,midpoints,mimewritable,min,min_fast,minabs,minimum,minimum!,minmax,minmax_fast,missing,mkdir,mkpath,mktemp,mktempdir,mod,mod1,mod2pi,modf,module_name,module_parent,mtime,muladd,mv,names,nb_available,ncodeunits,ndigits,ndims,next,nextfloat,nextind,nextpow,nextpow2,nextprod,nnz,nonzeros,norm,normalize,normalize!,normpath,notify,ntoh,ntuple,nullspace,num,num2hex,numerator,nzrange,object_id,occursin,oct,oftype,one,ones,oneunit,open,operm,ordschur,ordschur!,pairs,parent,parentindexes,parentindices,parse,partialsort,partialsort!,partialsortperm,partialsortperm!,peakflops,permute,permute!,permutedims,permutedims!,pi,pinv,pipeline,plan_bfft,plan_bfft!,plan_brfft,plan_dct,plan_dct!,plan_fft,plan_fft!,plan_idct,plan_idct!,plan_ifft,plan_ifft!,plan_irfft,plan_rfft,pointer,pointer_from_objref,pop!,popdisplay,popfirst!,position,pow_fast,powermod,precision,precompile,prepend!,prevfloat,prevind,prevpow,prevpow2,print,print_shortest,print_with_color,println,process_exited,process_running,prod,prod!,produce,promote,promote_rule,promote_shape,promote_type,push!,pushdisplay,pushfirst!,put!,pwd,qr,qrfact,qrfact!,quantile,quantile!,quit,rad2deg,rand,rand!,randcycle,randcycle!,randexp,randexp!,randjump,randn,randn!,randperm,randperm!,randstring,randsubseq,randsubseq!,range,rank,rationalize,read,read!,readandwrite,readavailable,readbytes!,readchomp,readdir,readline,readlines,readlink,readstring,readuntil,real,realmax,realmin,realpath,recv,recvfrom,redirect_stderr,redirect_stdin,redirect_stdout,redisplay,reduce,reducedim,reenable_sigint,reim,reinterpret,reload,relpath,rem,rem2pi,repeat,replace,replace!,repmat,repr,reprmime,reset,reshape,resize!,rethrow,retry,reverse,reverse!,reverseind,rfft,rm,rol,rol!,ror,ror!,rot180,rotl90,rotr90,round,rounding,rowvals,rpad,rsearch,rsearchindex,rsplit,rstrip,run,scale!,schedule,schur,schurfact,schurfact!,search,searchindex,searchsorted,searchsortedfirst,searchsortedlast,sec,secd,sech,seek,seekend,seekstart,select,select!,selectperm,selectperm!,send,serialize,set_zero_subnormals,setdiff,setdiff!,setenv,setindex!,setprecision,setrounding,shift!,show,showall,showcompact,showerror,shuffle,shuffle!,sign,signbit,signed,signif,significand,similar,sin,sin_fast,sinc,sincos,sind,sinh,sinh_fast,sinpi,size,sizehint!,sizeof,skip,skipchars,skipmissing,sleep,slicedim,sort,sort!,sortcols,sortperm,sortperm!,sortrows,sparse,sparsevec,spawn,spdiagm,speye,splice!,split,splitdir,splitdrive,splitext,spones,sprand,sprandn,sprint,spzeros,sqrt,sqrt_fast,sqrtm,squeeze,srand,stacktrace,start,startswith,stat,std,stdm,step,stride,strides,string,stringmime,strip,strwidth,sub2ind,subtypes,success,sum,sum!,sumabs,sumabs2,summary,super,supertype,svd,svdfact,svdfact!,svdvals,svdvals!,sylvester,symdiff,symdiff!,symlink,systemerror,take!,takebuf_array,takebuf_string,tan,tan_fast,tand,tanh,tanh_fast,task_local_storage,tempdir,tempname,thisind,tic,time,time_ns,timedwait,to_indices,toc,toq,touch,trace,trailing_ones,trailing_zeros,transcode,transpose,transpose!,tril,tril!,triu,triu!,trues,trunc,truncate,trylock,tryparse,typeintersect,typejoin,typemax,typemin,unescape_string,union,union!,unique,unique!,unlock,unmark,unsafe_copy!,unsafe_copyto!,unsafe_load,unsafe_pointer_to_objref,unsafe_read,unsafe_store!,unsafe_string,unsafe_trunc,unsafe_wrap,unsafe_write,unshift!,unsigned,uperm,valtype,values,var,varinfo,varm,vcat,vec,vecdot,vecnorm,versioninfo,view,wait,walkdir,warn,which,whos,widemul,widen,withenv,workspace,write,xcorr,xor,yield,yieldto,zero,zeros,zip,broadcast_getindex,broadcast_indices,broadcast_setindex!,broadcast_similar,dotview,apropos,doc,countfrom,cycle,drop,enumerate,flatten,partition,product,repeated,rest,take,zip,get_creds!,with,calloc,errno,flush_cstdio,free,gethostname,getpid,malloc,realloc,strerror,strftime,strptime,systemsleep,time,transcode,dlclose,dlext,dllist,dlopen,dlopen_e,dlpath,dlsym,dlsym_e,find_library,adjoint,adjoint!,axpby!,axpy!,bkfact,bkfact!,chol,cholfact,cholfact!,cond,condskeel,copy_transpose!,copyto!,cross,det,diag,diagind,diagm,diff,dot,eig,eigfact,eigfact!,eigmax,eigmin,eigvals,eigvals!,eigvecs,factorize,getq,givens,gradient,hessfact,hessfact!,isdiag,ishermitian,isposdef,isposdef!,issuccess,issymmetric,istril,istriu,kron,ldltfact,ldltfact!,linreg,logabsdet,logdet,lq,lqfact,lqfact!,lu,lufact,lufact!,lyap,norm,normalize,normalize!,nullspace,ordschur,ordschur!,peakflops,pinv,qr,qrfact,qrfact!,rank,scale!,schur,schurfact,schurfact!,svd,svdfact,svdfact!,svdvals,svdvals!,sylvester,trace,transpose,transpose!,transpose_type,tril,tril!,triu,triu!,vecdot,vecnorm,html,latex,license,readme,isexpr,quot,show_sexpr,add,available,build,checkout,clone,dir,free,init,installed,pin,resolve,rm,setprotocol!,status,test,update,deserialize,serialize,blkdiag,droptol!,dropzeros,dropzeros!,issparse,nnz,nonzeros,nzrange,permute,rowvals,sparse,sparsevec,spdiagm,spones,sprand,sprandn,spzeros,catch_stacktrace,stacktrace,cpu_info,cpu_summary,free_memory,isapple,isbsd,islinux,isunix,iswindows,loadavg,total_memory,uptime,atomic_add!,atomic_and!,atomic_cas!,atomic_fence,atomic_max!,atomic_min!,atomic_nand!,atomic_or!,atomic_sub!,atomic_xchg!,atomic_xor!,nthreads,threadid,applicable,eval,fieldtype,getfield,invoke,isa,isdefined,nfields,nothing,setfield!,throw,tuple,typeassert,typeof,uninitialized},%
    % types and modules
    keywords=[2]{AbstractArray,AbstractChannel,AbstractDict,AbstractDisplay,AbstractFloat,AbstractIrrational,AbstractMatrix,AbstractRNG,AbstractRange,AbstractSerializer,AbstractSet,AbstractSparseArray,AbstractSparseMatrix,AbstractSparseVector,AbstractString,AbstractUnitRange,AbstractVecOrMat,AbstractVector,Adjoint,Any,ArgumentError,Array,AssertionError,Bidiagonal,BigFloat,BigInt,BitArray,BitMatrix,BitSet,BitVector,Bool,BoundsError,BufferStream,CapturedException,CartesianIndex,CartesianIndices,Cchar,Cdouble,Cfloat,Channel,Char,Cint,Cintmax_t,Clong,Clonglong,Cmd,CodeInfo,Colon,Complex,ComplexF16,ComplexF32,ComplexF64,CompositeException,Condition,ConjArray,ConjMatrix,ConjVector,Cptrdiff_t,Cshort,Csize_t,Cssize_t,Cstring,Cuchar,Cuint,Cuintmax_t,Culong,Culonglong,Cushort,Cvoid,Cwchar_t,Cwstring,DataType,DenseArray,DenseMatrix,DenseVecOrMat,DenseVector,Diagonal,Dict,DimensionMismatch,Dims,DivideError,DomainError,EOFError,EachLine,Enum,Enumerate,ErrorException,Exception,ExponentialBackOff,Expr,Factorization,Float16,Float32,Float64,Function,GlobalRef,GotoNode,HTML,Hermitian,IO,IOBuffer,IOContext,IOStream,IPAddr,IPv4,IPv6,IndexCartesian,IndexLinear,IndexStyle,InexactError,InitError,Int,Int128,Int16,Int32,Int64,Int8,Integer,InterruptException,InvalidStateException,Irrational,KeyError,LabelNode,LinSpace,LineNumberNode,LinearIndices,LoadError,LowerTriangular,MIME,Matrix,MersenneTwister,Method,MethodError,MethodTable,Missing,MissingException,Module,NTuple,NamedTuple,NewvarNode,Nothing,Number,ObjectIdDict,OrdinalRange,OutOfMemoryError,OverflowError,Pair,PartialQuickSort,PermutedDimsArray,Pipe,Ptr,QuoteNode,RandomDevice,Rational,RawFD,ReadOnlyMemoryError,Real,ReentrantLock,Ref,Regex,RegexMatch,RoundingMode,RowVector,SSAValue,SegmentationFault,SerializationState,Set,Signed,SimpleVector,Slot,SlotNumber,Some,SparseMatrixCSC,SparseVector,StackFrame,StackOverflowError,StackTrace,StepRange,StepRangeLen,StridedArray,StridedMatrix,StridedVecOrMat,StridedVector,String,StringIndexError,SubArray,SubString,SymTridiagonal,Symbol,Symmetric,SystemError,TCPSocket,Task,Text,TextDisplay,Timer,Transpose,Tridiagonal,Tuple,Type,TypeError,TypeMapEntry,TypeMapLevel,TypeName,TypeVar,TypedSlot,UDPSocket,UInt,UInt128,UInt16,UInt32,UInt64,UInt8,UndefRefError,UndefVarError,UniformScaling,Uninitialized,Union,UnionAll,UnitRange,Unsigned,UpperTriangular,Val,Vararg,VecElement,VecOrMat,Vector,VersionNumber,WeakKeyDict,WeakRef,BLAS,Base,Broadcast,DFT,Docs,Iterators,LAPACK,LibGit2,Libc,Libdl,LinAlg,Markdown,Meta,Operators,Pkg,Serializer,SparseArrays,StackTraces,Sys,Threads,Core,Main},%
    % literals
    keywords=[1]{true,false,nothing,missing,im,uninitialized,NaN,NaN16,NaN32,NaN64,Inf,Inf16,Inf32,Inf64,ARGS,C_NULL,ENDIAN_BOM,ENV,LOAD_PATH,PROGRAM_FILE,STDERR,STDIN,STDOUT,VERSION},
    % keywords
    keywords=[1]{mutable,immutable,struct,begin,end,function,macro,quote,let,local,global,const,abstract,module,baremodule,using,import,export,in,if,else,elseif,for,while,do,try,type,catch,finally,return,break,continue},%
    sensitive=true,
    morecomment=[l]{\#},
    morecomment=[n]{\#=}{=\#},
    morestring=[s]{"}{"},
    morestring=[m]{'}{'},
    alsoletter=!?
}

\lstdefinestyle{julia}{
    backgroundcolor  = \color[HTML]{F2F2F2},
    basicstyle       = \ttfamily\footnotesize\color[HTML]{19177C},
    numberstyle      = \ttfamily\scriptsize\color[HTML]{7F7F7F},
    keywordstyle     = [1]{\bfseries\color[HTML]{1BA1EA}},
    keywordstyle     = [2]{\color[HTML]{0F6FA3}},
    keywordstyle     = [3]{\color[HTML]{0000FF}},
    stringstyle      = \ttfamily\color[HTML]{F5615C},
    commentstyle     = \color[HTML]{AAAAAA},
    rulecolor        = \color[HTML]{000000},
    frame=lines,
    xleftmargin=15pt,
    framexleftmargin=15pt,
    framextopmargin=4pt,
    framexbottommargin=4pt,
    tabsize=4,
    captionpos=b,
    breaklines=true,
    breakatwhitespace=false,
    showstringspaces=false,
    showspaces=false,
    showtabs=false,
    columns=fullflexible,
    keepspaces=true,
    numbers=none
}


\lstdefinelanguage{JuliaLocal}{
    language = Julia, % inherit Julia lang. to add keywords
    morekeywords = [3]{reset, initialize, generate_input, evaluate, bayesian_safety_validation, falsification, most_likely_failure, p_estimate}, % define more functions
    morekeywords = [2]{BayesianSafetyValidation, SystemParameters, Input, RunwayDetectionSystemParameters, OperationalParameters, TruncatedNormal, Normal}, % define more types and modules
}


%%%%%%%%%%%%%%%%%%%%%%%%%%%%%%%%%%%%%%%%
%% Formatting Helpers
%%%%%%%%%%%%%%%%%%%%%%%%%%%%%%%%%%%%%%%%
\newcommand{\rulesep}{\unskip\ \textcolor{lightgray}{\vrule}\ }
\newcommand{\tworow}[1]{\multirow{2}{*}{{#1}}}
\newcommand{\cpomcpow}{\cite{sunberg2018online,wu2021adaptive}}
\newcommand{\cdespot}{\cite{ye2017despot,wu2021adaptive}}
\newcommand{\cadaops}{\cite{wu2021adaptive}}
\newcommand{\scs}[1]{{\scriptsize#1}}
\newcommand{\pp}{\phantom{)}}
\newcommand{\td}[1]{\color{magenta}#1}
\newcommand{\negphantom}[1]{\settowidth{\dimen0}{#1}\hspace*{-\dimen0}}
\newcommand{\np}{\negphantom{)}}
\newcommand{\lit}[1]{{\color[rgb]{0.7,0.7,0.7}(#1)}\np}
\newcommand{\litdesc}[1]{\lit{#1}\phantom{)}}
\newcommand{\textdoublequotes}{\textsf{''}}
\newcommand{\sameresultsparen}{\textdoublequotes{}\phantom{\ 0.00)}}
\newcommand{\sameresults}{\textdoublequotes{}}


%%%%%%%%%%%%%%%%%%%%%%%%%%%%%%%%%%%%%%%%
%% TiKZ, PGF plots, and colors
%%%%%%%%%%%%%%%%%%%%%%%%%%%%%%%%%%%%%%%%
\usepackage{tikz}
\usepackage{pgfplots}
\pgfplotsset{compat=1.17}

\usetikzlibrary{calc}
\usetikzlibrary{shapes.geometric}
\usetikzlibrary{external}
\usetikzlibrary{patterns}
\usetikzlibrary{shapes,arrows,fit}
\usetikzlibrary{positioning}
\usetikzlibrary{arrows.meta, calc, shapes}
\usetikzlibrary{graphs}
\usetikzlibrary{decorations.pathmorphing}
\usetikzlibrary{decorations.pathreplacing}
\usetikzlibrary{backgrounds}
\usetikzlibrary{shadows}

\definecolor{blues1}{RGB}{198, 219, 239}
\definecolor{blues2}{RGB}{158, 202, 225}
\definecolor{blues3}{RGB}{107, 174, 214}
\definecolor{blues4}{RGB}{49, 130, 189}
\definecolor{blues5}{RGB}{8, 81, 156}

\definecolor{grays1}{RGB}{219, 219, 219}
\definecolor{grays2}{RGB}{202, 202, 202}
\definecolor{grays3}{RGB}{174, 174, 174}
\definecolor{grays4}{RGB}{130, 130, 130}
\definecolor{grays5}{RGB}{81, 81, 81}

\definecolor{gray1}{HTML}{999999}
\definecolor{gray2}{HTML}{CCCCCC}
\definecolor{gray3}{HTML}{D9D9D9}
\definecolor{gray4}{HTML}{EFEFEF}

\definecolor{darkgreen}{HTML}{38761D}
\definecolor{darkred}{HTML}{980000}
\definecolor{cardinal}{HTML}{B83A4B}

\definecolor{newcolor}{HTML}{C70039}
\definecolor{tablecolorgood}{HTML}{007662}
\definecolor{tablecolorbad}{HTML}{8c1515}
\definecolor{backgroundgray}{HTML}{F2F2F2}

\definecolor{timingcolor}{rgb}{0.55, 0.55, 0.55}
\newcommand{\tcolor}[1]{\color{timingcolor}#1}

% Standard colors
\def\primarycolor{black}
\def\secondarycolor{white}

%% PGFPlotsX.jl
\usetikzlibrary{arrows.meta}
\usetikzlibrary{backgrounds}
\usepgfplotslibrary{patchplots}
\usepgfplotslibrary{fillbetween}
\pgfplotsset{%
    layers/standard/.define layer set={%
        background,axis background,axis grid,axis ticks,axis lines,axis tick labels,pre main,main,axis descriptions,axis foreground%
    }{
        grid style={/pgfplots/on layer=axis grid},%
        tick style={/pgfplots/on layer=axis ticks},%
        axis line style={/pgfplots/on layer=axis lines},%
        label style={/pgfplots/on layer=axis descriptions},%
        legend style={/pgfplots/on layer=axis descriptions},%
        title style={/pgfplots/on layer=axis descriptions},%
        colorbar style={/pgfplots/on layer=axis descriptions},%
        ticklabel style={/pgfplots/on layer=axis tick labels},%
        axis background@ style={/pgfplots/on layer=axis background},%
        3d box foreground style={/pgfplots/on layer=axis foreground},%
    },
}


%%%%%%%%%%%%%%%%%%%%%%%%%%%%%%%%%%%%%%%%
%% Math
%%%%%%%%%%%%%%%%%%%%%%%%%%%%%%%%%%%%%%%%
\DeclareMathOperator{\Var}{Var}
\newcommand*{\defeq}{\stackrel{\text{def}}{=}}
\newcommand{\bpi}{\boldsymbol\pi}
\newcommand*\diff{\mathop{}\!\mathrm{d}}

\definecolor{resultcolor}{HTML}{000000}

\newsavebox\CBox
\def\mathBF#1{\sbox\CBox{$#1$}\resizebox{\wd\CBox}{\ht\CBox}{{\color{resultcolor}$\mathbf{#1}$}}}

\usepackage{siunitx}
\sisetup{detect-weight=true,detect-family=true,group-digits=false,separate-uncertainty,group-separator={,}}
\newcommand{\pmm}{{\normalfont$\pm$}}

% https://tex.stackexchange.com/questions/239242/why-does-big-mid-not-work
\makeatletter
\let\amsmath@bigmidtmp\bigm

\newcommand{\bigmid}[1]{%
  \ifcsname fenced@\string#1\endcsname
    \expandafter\@firstoftwo
  \else
    \expandafter\@secondoftwo
  \fi
  {\expandafter\amsmath@bigmidtmp\csname fenced@\string#1\endcsname}%
  {\amsmath@bigmidtmp#1}%
}
\newcommand{\DeclareFence}[2]{\@namedef{fenced@\string#1}{#2}}
\makeatother

\DeclareFence{\mid}{|}

% https://mirror.mwt.me/ctan/fonts/mathabx/texinputs/mathabx.sty
\DeclareFontFamily{U}{mathx}{\hyphenchar\font45}
\DeclareFontShape{U}{mathx}{m}{n}{
      <5> <6> <7> <8> <9> <10>
      <10.95> <12> <14.4> <17.28> <20.74> <24.88>
      mathx10
      }{}
\DeclareSymbolFont{mathx}{U}{mathx}{m}{n}
\DeclareFontSubstitution{U}{mathx}{m}{n}

% https://mirror.math.princeton.edu/pub/CTAN/fonts/mathabx/texinputs/mathabx.dcl
\DeclareMathSymbol{\betterbigvee}         {1}{mathx}{"9A}
\DeclareMathSymbol{\betterbigwedge}       {1}{mathx}{"99}

\DeclareMathAccent{\widebar}{0}{mathx}{"73} % from mathabx

\newcommand{\overarrow}[2]{\overset{\mathclap{\substack{#2 \\ \downarrow}}}{#1}}
\newcommand{\underarrow}[2]{\underset{\mathclap{\substack{\uparrow \\ #2}}}{#1}}

\let\oldvec\vec
\renewcommand{\vec}[1]{\vect{#1}}


%%%%%%%%%%%%%%%%%%%%%%%%%%%%%%%%%%%%%%%%
%% Commands
%%%%%%%%%%%%%%%%%%%%%%%%%%%%%%%%%%%%%%%%
\newcommand{\note}[1]{\textbf{\textcolor{cyan}{#1}}}
\newcommand{\citeneeded}{\textbf{\textcolor{magenta}{[CITE]}}}
\newcommand{\colorurl}[1]{\textcolor{cardinal}{\url{#1}}}

\newcommand{\new}[1]{{\color{newcolor}#1}}
\newcommand{\bpm}{\boldsymbol\pm}

% Switch between regular (symbol) and numeric footnotes
\renewcommand{\thefootnote}{\fnsymbol{footnote}}
\newcommand{\numericfootnote}[1]{%
\let\oldthefootnote=\thefootnote%
\stepcounter{mpfootnote}%
\addtocounter{footnote}{-1}%
\renewcommand{\thefootnote}{\arabic{mpfootnote}}%
\footnote{#1}%
\let\thefootnote=\oldthefootnote%
}

\setlength{\fboxsep}{0pt}
\setlength{\fboxrule}{0.3pt}
