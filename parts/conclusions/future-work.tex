\chapter{Future Work}\label{ch:future}
\chapterquote{Gas will expand to fill available space, work will expand to fill available time.}{Keith Schwarz}

This thesis presented research to address safe planning under uncertainty and efficient safety validation, but as with all research, it is only the beginning.
This chapter outlines several limitations of our work, possible next steps, and potential future research directions.

\paragraph{Neural network verification.}
Despite the convergence guarantees of adaptive conformal inference in ConstrainedZero \cite{gibbs2021adaptive}, planning with approximations does not provide safety guarantees.
Although our experiments suggest that adaptation helps find safe policies, studying formal methods for neural network verification is important \cite{liu2021algorithms}.
Applying nonlinear reachability methods such as Taylor models, conservative linearization, or partitioning \cite{althoff2021set,makino2003taylor,everett2021reachability,valbook} could help make the surrogates more robust.
Work from \textcite{katz2017reluplex} could be applied to simplified neural network surrogates used in this work.
Verification provides guarantees and an additional level of analysis complementary to validation, and should be studied before deploying neural networks into safety-critical systems.

\paragraph{Interpretable surrogate models.}
Especially in the safety case, having \textit{interpretable} surrogate models \cite{bougie2023interpretable} can be useful for stakeholders to better understand and analyze the safe decision making of the system.
An important question to research is: \textit{``How can interpretable surrogate models be used for safe planning?''}
Interpretable models, such as shallow decision trees \cite{de2013decision}, may be of interest.
When replacing complex surrogates like neural networks with simpler, more interpretable models, there tends to be a trade-off between interpretability (i.e., model complexity) and performance.
Studying this trade-off in the safety case is incredibly important.

\paragraph{Offline chance-constrained methods.}
Learning offline policies that guarantee satisfaction of safety constraints is another challenge that could be studied.
\textcite{ono2015chance} developed a method to solve chance-constrained MDPs given a target level of safety, yet their work could be extended to POMDPs and to multiple constraints.
Motivated by the ACAS~X safety case, \textcite{kochenderfer2012next} used QMDP \cite{littman1995learning} to train an offline policy that is augmented online with additional information.
Providing offline policy guarantees can help establish the safety case in fields such as aviation.
Such policies can then be verified to exhaust all combinations of a value-based lookup table \cite{jeannin2015formal}.

\paragraph{Belief representations.}
Extensions to algorithms such as BetaZero and ConstrainedZero could focus on addressing the belief representation as input to the neural network surrogates.
Future work could use the latent representation from deep conditional generative models like the $\mathcal{I}$-VAE introduced in \cref{ch:ivae}.
Other work, such as flow-based models \cite{chen2022flow}, principle component analysis \cite{roy2005finding}, or order-invariant network layers \cite{zaheer2017deep}, could also be studied as methods to represent the complex belief distributions.

\paragraph{Continuous action spaces.}
This thesis primarily deals with problems that have discrete action spaces.
Yet, more complicated systems may operate over \textit{continuous} actions.
Therefore, studying methods to extend our work to the continuous action domain would be useful.
\textcite{moerland2018a0c} introduced an extension to AlphaZero that handles continuous actions, and their work could be directly applied to the BetaZero and ConstrainedZero algorithms.
Future work could therefore study how to apply our methods to safety-critical robotics tasks such as control \cite{sylvie2010planning} and object manipulation \cite{pajarinen2017robotic,pajarinen2022pomdp}.

\paragraph{Surrogates for environment models.}
Another avenue of future work is the use of surrogate models to replace expensive geological simulators, such as those used by \textcite{wen2021ccsnet} for the CCS problem.
Research into approximate \textit{world models} \cite{ha2018world} could be directly applicable to our work and provide additional computational gains.
Therefore, it is important to study how best to represent the state-transition dynamics using surrogate models.
